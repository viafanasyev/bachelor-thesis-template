Пример приложения. Какой-то текст. Какой-то текст. Какой-то текст. Какой-то текст. Какой-то текст. Какой-то текст. Какой-то текст. Какой-то текст. Какой-то текст. Какой-то текст. Какой-то текст. Какой-то текст.

Тут ссылка на листинг~\ref{kotlin:1}.

А тут ссылка на листинг~\ref{bytecode:1}.

\begin{lstlisting}[language=kotlin,moredelim={**[is][\btHL]{[@}{@]}},escapeinside={`}{`},caption={Пример какого-то кода на Kotlin},label=kotlin:1]
@Deprecated("Reason")
fun findScriptDefinition(project: Project, script: SourceCode): ScriptDefinition? {
  val scriptDefinitionProvider = ScriptDefinitionProvider.getInstance(project) ?: [@return null@]
  ``[@?:@] throw IllegalStateException("Unable to get script definition: ...")

  return scriptDefinitionProvider.findDefinition(script) ?: scriptDefinitionProvider.getDefaultDefinition() // Comment
}
\end{lstlisting}

\begin{lstlisting}[language=java,caption={Пример какого-то кода на Java},label=java:1]
class Main {
  public static ScriptDefinition findScriptDefinition(Project project, SourceCode script) {
    ScriptDefinitionProvider scriptDefinitionProvider = ScriptDefinitionProvider.getInstance(project);
    if (scriptDefinitionProvider == null) {
      if (null == null) {
        throw IllegalStateException("Unable to get script definition: ...");
      } else {
        return null;
      }
    }

    ScriptDefinition definition = scriptDefinitionProvider.findDefinition(script);
    if (definition == null) {
      return scriptDefinitionProvider.getDefaultDefinition(); // Comment
    } else {
      return definition;
    }
  }
}
\end{lstlisting}

\newcommand{\createlinenumber}[2]{
    \edef\thelstnumber{%
        \unexpanded{%
            \ifnum#1=\value{lstnumber}\relax
              #2%
            \fi%
        }%
        \ifx\thelstnumber\relax%
        \else\expandafter\unexpanded\expandafter{\thelstnumber}%
        \fi%
    }
}

\bgroup
\let\thelstnumber\relax
\createlinenumber{2}{13}
\createlinenumber{3}{14}
\createlinenumber{4}{15}
\createlinenumber{5}{18}
\createlinenumber{6}{21}
\createlinenumber{7}{22}
\createlinenumber{8}{24}
\createlinenumber{9}{27}

\createlinenumber{11}{46}
\createlinenumber{12}{47}
\createlinenumber{13}{48}
\createlinenumber{14}{51}
\createlinenumber{15}{54}
\createlinenumber{16}{55}
\createlinenumber{17}{57}
\createlinenumber{18}{60}
\begin{lstlisting}[language=JVMIS,moredelim={**[is][\btHL]{[@}{@]}},escapeinside={`}{`},caption={Пример JVM-байткода},label=bytecode:1,numbers=left]
...
aload_2
dup
[@ifnonnull     28@]
new           #17 // NullPointerException
dup
ldc           #19 // String null cannot be cast to non-null String
invokespecial #23 // NullPointerException."<init>"(String)
athrow
...
aload_2
dup
[@ifnonnull     61@]
new           #17 // NullPointerException
dup
ldc           #19 // String null cannot be cast to non-null String
invokespecial #23 // NullPointerException."<init>"(String)
athrow
...
\end{lstlisting}
\egroup

\begin{lstlisting}[language=JVMIS,moredelim={**[is][\btHL]{[@}{@]}},numbers=none,escapeinside={`}{`},caption={Пример JVM-байткода 2},label=bytecode:2]
...
13: aload_2
14: dup
[@15: ifnonnull     28@]
18: new           #17 // NullPointerException
21: dup
22: ldc           #19 // String null cannot be cast to non-null String
24: invokespecial #23 // NullPointerException."<init>"(String)
27: athrow
...
46: aload_2
47: dup
[@48: ifnonnull     61@]
51: new           #17 // NullPointerException
54: dup
55: ldc           #19 // String null cannot be cast to non-null String
57: invokespecial #23 // NullPointerException."<init>"(String)
60: athrow
...
\end{lstlisting}

А тут ссылка на таблицу~\ref{table:1}.

\begin{table}[h!]
\centering
\begin{tabular}{| c | c | c | c |} 
    \hline
    Col1 & Col2 & Col2 & Col3 \\ [0.5ex] 
    \hline
    1 & 6 & 87837 & 787 \\
    \hline
    2 & 7 & 78 & 5415 \\
    \hline
    3 & 545 & 778 & 7507 \\
    \hline
    4 & 545 & 18744 & 7560 \\
    \hline
    5 & 88 & 788 & 6344 \\
    \hline
\end{tabular}
\caption{Пример таблицы}
\label{table:1}
\end{table}

\begin{lstlisting}[style=algorithmic, caption={Привер псевдокода на алгоритмическом языке}]
procedure RUN($packages, hashes$)
    $queue[svace.parallel\_max]$
    for $item \in zip(packages,hashes) $
        $ps = create(item)$
        if $!queue.full()$
            $queue.put(ps)$
        else
            $first = queue.get()$
            $first.wait()$
        end if
    end for
end procedure
\end{lstlisting}
